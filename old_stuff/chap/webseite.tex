\section{Graphische Darstellung der Daten mit LabVIEW}
Für die Darstellung der ausgelesenen Daten wird ein \textbf{LabVIEW\footnote{\textbf{LabVIEW}: \textbf{Lab}oratory \textbf{V}irtual \textbf{I}nstrument \textbf{E}ngineering \textbf{W}orkbench, ist ein graphisches Programmiersystem von \textbf{National Instruments}. Die Programmierung erfolgt nach dem Datenfluss-Modell in der graphischen Programmiersprache \textbf{G}.}-Programm} geschrieben.
Zur Kommunikation mit Datenbanken stellt das \textbf{LabVIEW Database Connectivity Toolkit} einige Programme bzw. Funktionen bereit. Diese nutzen \textbf{ODBC} um auf die Daten zuzugreifen. Das Programm bietet die Möglichkeit einzustellen, aus welchem Zeitraum die Datensätze ausgelesen werden sollen.

%Um die ausgelesenen Daten darzustellen, wird ein \textbf{LabVIEW}-Programm erstellt.
%...bla bla labview database connectivity toolkit. das kann odbc nutzen. mit select alle %datensätze zwischen a und b besorgen... sowas halt. mehr ins detail gehen lohnt sich dann, %wenn das programm mal fertig ist



% "Schaltung" einfügen
%Screenshot einfügen
\section{php}