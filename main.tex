\documentclass{include/thesisclass}
% Main File - Based on thesisclass.cls
% Comments are mostly in English
% ------------------------------------------------------------------------------
% Further files in folder:
%  - include/cmds.tex (for macros and additional commands)
%  - include/kitlogo.pdf (for titlepage)
%  - lit.bib (bibtex bibliography database)
%  - include/titlepage.tex (for layout of titelpage)
% ------------------------------------------------------------------------------
% Useful Supplied Packages:
% amsmath, amssymb, mathtools, bbm, upgreek, nicefrac,
% siunitx, varioref, booktabs, graphicx, tikz, multicol

\usepackage{etoolbox} % for "\patchcmd" macro
\makeatletter
% No extra space between chapter number and chapter header lines:
\patchcmd{\@makechapterhead} {\vskip 20}{\vskip 0} {}{}
% Reduce extra space between chapter header and section header lines by 50%:
\patchcmd{\@makechapterhead} {\vskip 40}{\vskip 20}{}{}
\patchcmd{\@makeschapterhead}{\vskip 40}{\vskip 20}{}{} % for unnumbered chapters
\makeatother

\usepackage{standalone}

\usepackage{listings}
\lstdefinestyle{DOS}
{
    backgroundcolor=\color{black},
    basicstyle=\scriptsize\color{white}\ttfamily
}

\newcommand{\shellcmd}[1]{\\\indent\indent\texttt{\footnotesize\# #1}\\}

\usepackage{rotating}
\usepackage{epsfig}
\usepackage{caption}
\usepackage{subcaption}
\usepackage{graphicx}
\usepackage{epsfig,rotating,amsmath}
%\usepackage[numbers]{natbib}
\usepackage{hyperref}
%\usepackage[3D]{movie15}
\usepackage{siunitx}
\usepackage{calc}
\usepackage{float}
\usepackage{mathtools}

\usepackage{tabularx}
\newcolumntype{L}{>{\raggedright\arraybackslash}X}

\usepackage{booktabs}
\newcolumntype{C}{>{\Centering\arraybackslash}X} % centered "X" column

\usepackage{lscape}
\newcommand{\properparagraph}[1]{\paragraph{\textcolor{black}{#1}}\mbox{}\\}

%%%%%%%%%%%%%%%% Tikz %%%%%%%%%%%%%%%%%
\usepackage{tikz,tikzscale,pgfplots,pgfplotstable,circuitikz}
\usepackage{flowchart}
\usetikzlibrary{shapes.geometric}
\usetikzlibrary{arrows.meta}
\usetikzlibrary{arrows}
\usetikzlibrary{shapes.symbols,shadows}

%%%%%%%%%%%%%%%%%%%%%%%%%%%%%%%%%%%%%%

\usepackage{palatino}

%%% Abkürzungverzeichnis %%%
\usepackage{nomencl}
\let\abbrev\nomenclature

% Abkürzungsverzeichnis LiveTex Version
\renewcommand{\nomname}{Abkürzungsverzeichnis}
\setlength{\nomlabelwidth}{.25\hsize}
\renewcommand{\nomlabel}[1]{#1 \dotfill}
\setlength{\nomitemsep}{-\parsep}
\makenomenclature
%\makeglossary
%\usepackage{bmpsize}

%% -------------------------
%% |    Thesis Settings    |
%% -------------------------
% english or ngerman (new german für neue deutsche Rechtschreibung statt german)
\SelectLanguage{ngerman}
% details on this thesis
\newcommand{\thesisauthor}{Olena Manzhura, Marvin Noll}
\newcommand{\thesistopic}{Dokumentation IEH-Wetterstation}
%\newcommand{\thesisentopic}{Name of the Topic in English}
%\newcommand{\thesislongtopic}{Very long and very detailed description of the very interesting thesis topic (only necessary for pdfsubject tag).}
\newcommand{\thesisinstitute}{Institut für Elektroenergiesysteme und Hochspannungstechnik}
\newcommand{\thesisreviewerone}{Sebastian Hubschneider, Wolf Schulze}
\newcommand{\thesisadvisorone}{} % to use: enter names and uncomment in titlepg
%\newcommand{\thesispagehead}{Bachelor Thesis: \thesistopic} % page heading





%% ---------------------
%% |    PDF - Setup    |
%% ---------------------
% This information will appear embed into the PDF file as meta data, but will 
% not be printed anywhere
%\hypersetup
%{
%    pdfauthor={\thesisauthor},
%    pdftitle={Bachelorarbeit: \thesistopic},
%    pdfsubject={\thesislongtopic},
%    pdfkeywords={kit,elektrotechnik,bachelor,thesis,\thesisauthor}
%}





%% --------------------------------------
%% |    Settings for Word Separation    |
%% --------------------------------------
% Help for separation:
% In German package the following hints are additionally available:
% "- = Additional separation
% "| = Suppress ligation and possible separation (e.g. Schaf"|fell)
% "~ = Hyphenation without separation (e.g. bergauf und "~ab)
% "= = Hyphenation with separation before and after
% "" = Separation without a hyphenation (e.g. und/""oder)

% Describe separation hints here:
\hyphenation
{
    über-nom-me-nen an-ge-ge-be-nen
    %Pro-to-koll-in-stan-zen
    %Ma-na-ge-ment  Netz-werk-ele-men-ten
    %Netz-werk Netz-werk-re-ser-vie-rung
    %Netz-werk-adap-ter Fein-ju-stier-ung
    %Da-ten-strom-spe-zi-fi-ka-tion Pa-ket-rumpf
    %Kon-troll-in-stanz
}

\usepackage[citestyle=numeric,style=numeric,backend=biber]{biblatex}
\addbibresource{lit.bib}
\raggedbottom

\begin{document}
    % Titlepage and ToC
    \FrontMatter
    \input{include/titlepage}
    
    \begingroup    % in order to avoid listoffigures and
    \tableofcontents                    % listoftables on new pages
    \listoffigures
    \listoftables
    \endgroup
    %\cleardoublepage
    
    \MainMatter
    \chapter{Inbetriebnahme}
\properparagraph{Raspberry Pi}
\properparagraph{DebianVM}
    \chapter{Allgemeine Struktur}
Bla ganz toll die einzelnen Sachen erklären. Noch ein Blockdiagramm einfügen.
\section{Loggen der Daten}
Der Code befindet sich unter \href{https://github.com/soerman/sensordataToMQTT}{https://github.com/soerman/sensordataToMQTT} heruntergeladen werden. Im Folgenden werden kurz die einzelnen Module bzw. Klassen beschrieben.

\properparagraph{Main}
\properparagraph{dlxMetDatalogger}
\properparagraph{measurement}
\input{chap/general/grafana/grafana}    

    \Appendix
    \chapter*{\appendixname} \addcontentsline{toc}{chapter}{\appendixname}
    \section{Datalogger DLx-MET}

In \autoref{tbl:anschluesse} sind die Sensoren der Wetterstation und deren jeweiliger Anschluss aufgeführt. In dieser Reihenfolge befinden sich die Daten auch im "'Array"', der geschickt wird, nachdem man einen Auslese-Befehl gesendet hat.

\begin{table}[H]
	\centering
	\caption{Anschlüsse der Sensoren an den Datenlogger}
	\label{tbl:anschluesse}
	\begin{tabularx}{0.5\textwidth}{XX}  
		\toprule
		\textbf{Sensor} & \textbf{Beschreibung} \\
		\midrule
		1 	&  Windgeschwindigkeit      	\\
		2 	&  Windrichtung 	    \\
		3   &	Temperatur 1 \\
		4   & relative Feuchtigkeit\\
		5   & Niederschlag  \\
		6   & Luftdruck \\
		7   & Einstrahlung \\
		8   & Temperatur 2\\
		\bottomrule
	\end{tabularx}
\end{table}


    %\chapter*{\appendixname} \addcontentsline{toc}{chapter}{\appendixname}
    \addcontentsline{toc}{section}{Abkürzungsverzeichnis}
    % das Abkürzungsverzeichnis entgültige Ausgeben
    \fancyhead[L]{Abkürzungsverzeichnis} %Kopfzeile links
    \newpage

    % Bibliography
    \nocite{*}
    \printbibliography[heading=bibintoc]
    %\bibliographystyle{alpha}
    %\bibliography{lit}
    % BIBTEX
    % use if you want citations to appear even if they are not referenced to: 
    % \nocite{*} or maybe \nocite{Kon64,And59} for specific entries
    %\nocite{*}

    % THEBIBLIOGRAPHY
    %\begin{thebibliography}{000}
    %    \bibitem{ident}Entry into Bibliography.
    %\end{thebibliography}
\end{document}
