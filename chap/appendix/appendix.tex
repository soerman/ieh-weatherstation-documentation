\section{Arbeiten mit Containern}
Um im Container Kommandos auszuführen/zu arbeiten, kann mit dem Befehl
\begin{verbatim}
docker exec -it [container-id] bash
\end{verbatim}
eine Bash-Shell im entsprechenden Container gestartet werden. Die Container-IDs der aktiven Container können mit 
\begin{verbatim}
docker ps
\end{verbatim}
aufgelistet werden.

\section{Datalogger DLx-MET}

In \autoref{tbl:anschluesse} sind die Sensoren der Wetterstation und deren jeweiliger Anschluss aufgeführt. In dieser Reihenfolge befinden sich die Daten auch im "'Array"', der geschickt wird, nachdem man einen Auslese-Befehl gesendet hat.

\begin{table}[H]
	\centering
	\caption{Anschlüsse der Sensoren an den Datenlogger}
	\label{tbl:anschluesse}
	\begin{tabularx}{0.5\textwidth}{XX}  
		\toprule
		\textbf{Sensor} & \textbf{Beschreibung} \\
		\midrule
		1 	&  Windgeschwindigkeit      	\\
		2 	&  Windrichtung 	    \\
		3   &	Temperatur 1 \\
		4   & relative Feuchtigkeit\\
		5   & Niederschlag  \\
		6   & Luftdruck \\
		7   & Einstrahlung \\
		8   & Temperatur 2\\
		\bottomrule
	\end{tabularx}
\end{table}

