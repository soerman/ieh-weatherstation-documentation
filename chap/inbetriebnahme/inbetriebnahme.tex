\chapter{Inbetriebnahme}
Im Folgenden werden die notwendigen (Installations-)Schritte zur Inbetriebnahme des Gesamtsystems auf neu installierten Betriebssystemen beschrieben.
\section{DebianVM}
Vor dem Installieren der Pakete ein Update und Upgrade durchführen:
\begin{verbatim}
$ sudo apt-get -y update 
$ sudo apt-get -y upgrade
$ sudo apt -y autoremove
\end{verbatim}

\subsection{git}
\texttt{git} wird mit \verb|$ sudo apt install git| installiert.

\subsection{Docker und Docker-Compose}
Zur Installation von Docker und Docker-Compose auf Debian zu installieren, sind folgende Schritte durchzuführen\footnote{Siehe \href{https://docs.docker.com/engine/install/debian/}{https://docs.docker.com/engine/install/debian/}}
\begin{enumerate}
	\item Pakete installieren, um das Benutzen von Repos über HTTPS zu ermöglichen: 
	\begin{verbatim}
$ sudo apt-get install \
    apt-transport-https \
    ca-certificates \
    curl \
    gnupg-agent \
    software-properties-common
\end{verbatim}
	\item GPG Schlüssel von Docker hinzufügen:
	\begin{verbatim}
	$ curl -fsSL https://download.docker.com/linux/debian/gpg \
	| sudo apt-key add 
	\end{verbatim}
	\item Mit \verb|$ sudo apt-key fingerprint 0EBFCD88| den Fingerprint des Schlüssels überprüfen. Dieser sollte \texttt{9DC8 5822 9FC7 DD38 854A E2D8 8D81 803C 0EBF CD88} entsprechen.
	\item Repository hinzufügen:
	\begin{verbatim}
	$ sudo add-apt-repository \
	"deb [arch=amd64] https://download.docker.com/linux/debian \
	$(lsb_release -cs) \
	stable"
	\end{verbatim}
	\item Docker installieren:
	\begin{verbatim}
	 $ sudo apt-get update
	 $ sudo apt-get install docker-ce docker-ce-cli containerd.io
	\end{verbatim}
	\item Docker-Compose:
	\begin{verbatim}
	sudo apt install docker-compose
	\end{verbatim}
\end{enumerate}


\subsection{MQTT/Telegraf/InfluxDB/Grafana-Container}
Das Repo mit 
\begin{verbatim}
git clone 
https://github.com/youcann/mqttTelegrafInfluxGrafana-dockerCompose.git
\end{verbatim}
herunterladen. Ins Verzeichnis wechseln und mit \verb|docker-compose up| die Container starten.

\newpage

\section{Raspberry Pi}

\subsection{git}
\texttt{git} wird mit \verb|$ sudo apt install git| installiert.\newline
Im gewünschten Verzeichnis, z.B. im Home-Verzeichnis, das Repo mit 
\begin{verbatim}
	$ git clone https://github.com/soerman/sensordataToMQTT.git
\end{verbatim}
herunterladen. 

Vor dem Installieren der Pakete ein Update und Upgrade durchführen:
\begin{verbatim}
$ sudo apt-get -y update 
$ sudo apt-get -y upgrade
$ sudo apt -y autoremove
\end{verbatim}

\subsection{Feste IP-Adresse und log2ram einstellen}
In das Repo-Verzeichnis wechseln. Im Ordner \verb|bashScripts| das Skript zum Einstellen der festen IP-Adresse (172.23.62.255) und \texttt{log2ram} mit 
\begin{verbatim}
$ bash ip_log2ram.sh
\end{verbatim}
ausführen. Anschließend den Pi neustarten.




\subsection{Docker und Docker-Compose}
Um Docker und Docker-Compose zu installieren folgende Schritte ausführen\footnote{Siehe \href{https://dev.to/rohansawant/installing-docker-and-docker-compose-on-the-raspberry-pi-in-5-simple-steps-3mgl}{https://dev.to/rohansawant/installing-docker-and-docker-compose-on-the-raspberry-pi-in-5-simple-steps-3mgl}}:
\begin{enumerate}
	\item Docker installieren:
	\begin{verbatim}
		$ curl -sSL https://get.docker.com | sh
	\end{verbatim}
	\item Berechtigung für den Nutzer \texttt{pi} erteilen, Docker-Kommandos auszuführen:
	\begin{verbatim}
		$ sudo usermod -aG docker pi
		$ newgrp docker
	\end{verbatim}
	\item Test:
	\begin{verbatim}
		$ docker run hello-world
	\end{verbatim}
	\item Abhängigkeiten installieren:
	\begin{verbatim}
		$ sudo apt-get install -y libffi-dev libssl-dev
		$ sudo apt-get install -y python3 python3-pip
		$ sudo apt-get remove python-configparser
	\end{verbatim}
	\item Docker-Compose installieren:
	\begin{verbatim}
		$ sudo pip3 -v install docker-compose
	\end{verbatim}

\end{enumerate}

\subsection{Datenlogger-Skript}
Im Verzeichnis mit den Dateien mit \verb|$ docker-compose up --build| den Docker-Container starten.

\paragraph{Anpassung}
In der \verb|docker-compose.yaml| können folgende Punkte angepasst werden:
\begin{itemize}
	\item Port an dem die Wetterstation angeschlossen ist
	\item IP-Adresse des MQTT-Servers
	\item Port des MQTT-Servers
	\item MQTT-Topic
	\item Intervall (in Sekunden) mit dem Wetterdaten abgefragt werden
	\item Pfad zum Log-File
	\item Die Log-Level, bei denen die jeweiligen Python-Logger-Handler loggen
	\item Pfad zur Datei, die zum Watchdog-Reset beschrieben werden soll
\end{itemize} 

\subsection{Watchdog}
Watchdog erst ganz zum Schluss starten, wenn das Programm ohne Probleme gestartet hat/läuft!

Im Verzeichnis \verb|bashScripts| mit 
\begin{verbatim}
$ bash watchdogStart.sh
\end{verbatim}
das Skript ausführen um den Watchdog zu starten.


\section{Webseite}
Nach erfolgreicher Inbetriebnahme ist die Webseite unter \texttt{172.23.62.107:3000} aufrufbar mit den folgenden Zugangsdaten\footnote{Lassen sich bei Erstanmeldung ändern}

\begin{verbatim}
	Benutzername: admin
	Passwort: admin
\end{verbatim}
